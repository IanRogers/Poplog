



%                  ***************************************
%                  THE POPLOG REFERENCE MANUAL MASTER FILE
%                  ***************************************

% *******************************VOLUME ONE ********************************

%THIS IS THE TITLE PAGE INFORMATION - ALL THE FORMATTING INFORMATION DEALS
% ONLY WITH THE PLACING OF THIS TEXT.

\begin{titlepage}
\begin{center}
\rule{0cm}{3cm}
{\Huge\bf POPLOG REFERENCE \\
\vspace{0.5cm}
MANUAL}\\
\vspace{3cm}
{\bf\Large
Volume One }\\
\vspace{1cm}
{\LARGE\bf POPLOG STRUCTURE / POP-11}\\
\vspace{1cm}
{\bf\Large \today }\\
\vspace{10cm}
{\LARGE\bf Prepared and Formatted by \\ Diarmuid McIntyre}\\
\vspace{1.5cm}
{\Large Based on the On-Line REF files \\}
\end{center}

\end{titlepage}


% THIS SECTION IS THE CURRENT MANUAL INTRODUCTION. IF YOU WISH IT
% COULD BE INCLUDED INSTEAD OF BEING EXPLICITLY LOCATED IN THE
% MASTER FILE.

\rule{0cm}{3cm}
\begin{center}
{\Huge\bf Preface}
\end{center}
\vspace{1cm}

 The document you are holding in your hands has been automatically
generated from the on-line REF files by the REFORMAT program.

It has been produced by wrapping the text of those files in LATEX
formatting commands. If you have a Unix system (with LATEX) and are
running X, then it is possible for you to generate manuals which are
tailored to your needs by only including those REF files you have use
for. HELP REFORMAT and REF REFORMAT give details of this.


This Volume is one of four manuals (Volume 2 having been split into
two). You will find that page cross-references [in square brackets such
as these] are restricted to those items appearing in this manual only.
However if the phrase "(included in another Volume)" appears after a
reference to a REF file then you can look in the appropriate volume for
the REF file. This will appear according to its title not its name.
	Any bolded identifier names without a page reference can be found in
other Volumes.


Due to the fact that the manual is created completely automatically, the
Index deals only with identifiers. The Table of Contents should be
referenced when searching for a subject topic. This is generated from
section and subsection headings in the REF files as well as the titles
of the REF files (which form the chapter titles). The index is added to each
time an identifier name which is defined in the same volume appears.

	 In the Index, the bolded page number after an entry is that on
which the identifier is actually defined, as opposed to being mentioned
in conjunction with something else. These latter references appear in
normal print.


\section*{Acknowledgements}

The REFORMAT program was written in Pop-11 by Diarmuid McIntyre. The REF
files were written by the various members of the Poplog team over the
years. The names of the original writers (and revisors) of a REF file
appear as a chapters revision history. Special thanks go to Ben du
Boulay, Adrian Howard, John Williams, John Gibson, Rob Duncan and Simon
Nichols, all of the University of Sussex.


% THIS LINE PLACES THE TABLE OF CONTENTS (WHICH ARE LOCATED IN THE .TOC
% VERSION OF THIS FILE) INTO THE FILE AT THE APPROPRIATE PLACE

\tableofcontents

% THIS LINE CAUSES ARABIC NUMBERING TO BEGIN WITH THE TABLE OF CONTENTS
\pagenumbering{arabic}




% THE FOLLOWING LINE (AND OTHERS LIKE IT LATER!) DIVIDES THE MANUAL UP INTO
% GROUPS OF CHAPTERS. THIS IS THE START OF ONE SUCH GROUPING.
% NOTE THE UPPERCASE.

\part{HOW POPLOG IS IMPLEMENTED}


% THESE NEXT FEW LINES PROVIDE A SORT OF SUBGROUPING PREAMBLE TELLING THE
% READER OF THE FURTHER SUBDIVISION OF THE MANUAL. THIS ENABLES THE READER
% TO SEE WHAT SORT OF MATERIAL IS COVERED WITHOUT NEEDING TO REFER BACK TO
% THE TABLE OF CONTENTS.


\chapter*{}
\begin{flushright}
\vspace{1in}
\parbox{5.2in}{


This part of the manual is sub-divided into 3 sections:

\begin{itemize}
  \item Input, Output, and Operating Systems
  \item Run-time and the Virtual Machine
  \item The Structure of Poplog
\end{itemize}
}
\end{flushright}


\part*{Input, Output, and Operating Systems}

\refinclude{sysio}
\refinclude{chario}
\refinclude{itemise}
\refinclude{sysutil}
\refinclude{vms_sysutil}
\refinclude{signals}


\part*{Run-time and the Virtual Machine}


\refinclude{vmcode}
\refinclude{process}
\refinclude{procedure}
\refinclude{stack}


\part*{The Structure of Poplog}

\refinclude{system}
\refinclude{library}
\refinclude{subsystem}
\refinclude{x}


\part{DEFINING AND ACCESSING STRUCTURES}

\refinclude{defstruct}
\refinclude{external}
\refinclude{external_data}
\refinclude{newc_dec}


\part{THE POPLOG IMPLEMENTION OF POP-11}


\chapter*{}
\begin{flushright}
\vspace{1in}
\parbox{5.2in}{


This part of the manual is sub-divided into 2 sections:

\begin{itemize}
  \item Data Structures
  \item Syntax and Compilation
\end{itemize}
}
\end{flushright}


\part*{Data Structures}

\refinclude{data}
\refinclude{arrays}
\refinclude{lists}
\refinclude{strings}
\refinclude{words}
\refinclude{numbers}
\refinclude{vectors}
\refinclude{intvec}
\refinclude{keys}
\refinclude{props}



\part*{Syntax and Compilation}

\refinclude{syntax}
\refinclude{popcompile}
\refinclude{sections}
\refinclude{proglist}
\refinclude{ident}


\part{VARIOUS IMPLEMENTATION ASPECTS}

\refinclude{environment_variables}
\refinclude{trace}
\refinclude{print}
\refinclude{times}
\refinclude{mishaps}
\refinclude{mishap_codes}
\refinclude{documentation}
\refinclude{obsolete}
